
% Document class: article with font size 11pt
% ---------------
\documentclass[11pt,a4paper]{report}

\setlength{\textwidth}{165mm}
\setlength{\textheight}{240mm}
\setlength{\parindent}{0mm} % S{\aa} meget rykkes ind efter afsnit
\setlength{\parskip}{\baselineskip}
\setlength{\headheight}{0mm}
\setlength{\headsep}{0mm}
\setlength{\hoffset}{-2.5mm}
\setlength{\voffset}{0mm}
\setlength{\footskip}{15mm}
\setlength{\oddsidemargin}{0mm}
\setlength{\topmargin}{0mm}
\setlength{\evensidemargin}{0mm}

\usepackage[a4paper, hmargin={2.8cm, 2.8cm}, vmargin={2.5cm, 2.5cm}]{geometry}
\usepackage[super]{nth}
\PassOptionsToPackage{hyphens}{url}\usepackage{hyperref}
\usepackage{eso-pic} % \AddToShipoutPicture
\usepackage{float} % This will allow precise picture placement, use [H].


% Call packages
% ---------------
\usepackage{comment} %Possible to comment larger sections
%http://get-software.net/macros/latex/contrib/comment/comment.pdf
\usepackage[T1]{fontenc} %oriented to output, that is, what fonts to use for printing characters.
\usepackage[utf8]{inputenc} %allows the user to input accented characters directly from the keyboard
%Put fontenc before inputenc
\usepackage{fourier} %Better typeface
%http://mirrors.dotsrc.org/ctan/fonts/fourier-GUT/doc/latex/fourier/fourier-doc-en.pdf
\usepackage[english]{babel}														     % Danish
\usepackage[protrusion=true,expansion=true]{microtype}				                 % Better typography
%http://www.khirevich.com/latex/microtype/
\usepackage{amsmath,amsfonts,amsthm, amssymb}							 % Math packages
\usepackage[pdftex]{graphicx} %puts to pdf and graphic
%http://www.kwasan.kyoto-u.ac.jp/solarb6/usinggraphicx.pdf
\usepackage{xcolor,colortbl}
%http://mirrors.dotsrc.org/ctan/macros/latex/contrib/xcolor/xcolor.pdf
%http://texdoc.net/texmf-dist/doc/latex/colortbl/colortbl.pdf
\usepackage{tikz} %documentation http://www.ctan.org/pkg/pgf
\usepackage{parskip} %http://www.ctan.org/pkg/parskip
%http://tex.stackexchange.com/questions/51722/how-to-properly-code-a-tex-file-or-at-least-avoid-badness-10000
%Never use \\ but instead press "enter" twice. See second website for more info

% MATH -------------------------------------------------------------------
\newcommand{\Real}{\mathbb R}
\newcommand{\Complex}{\mathbb C}
\newcommand{\Field}{\mathbb F}
\newcommand{\RPlus}{[0,\infty)}
%
\newcommand{\norm}[1]{\left\Vert#1\right\Vert}
\newcommand{\essnorm}[1]{\norm{#1}_{\text{\rm\normalshape ess}}}
\newcommand{\abs}[1]{\left\vert#1\right\vert}
\newcommand{\set}[1]{\left\{#1\right\}}
\newcommand{\seq}[1]{\left<#1\right>}
\newcommand{\eps}{\varepsilon}
\newcommand{\To}{\longrightarrow}
\newcommand{\RE}{\operatorname{Re}}
\newcommand{\IM}{\operatorname{Im}}
\newcommand{\Poly}{{\cal{P}}(E)}
\newcommand{\EssD}{{\cal{D}}}
% THEOREMS ----------------------------------------------------------------
\theoremstyle{plain}
\newtheorem{thm}{Theorem}[section]
\newtheorem{cor}[thm]{Corollary}
\newtheorem{lem}[thm]{Lemma}
\newtheorem{prop}[thm]{Proposition}
%
\theoremstyle{definition}
\newtheorem{defn}{Definition}[section]
%
\theoremstyle{remark}
\newtheorem{rem}{Remark}[section]
%
\numberwithin{equation}{section}
\renewcommand{\theequation}{\thesection.\arabic{equation}}

\author{
\Large{
  Gram, Mads (\href{mailto:mgmadsgram@gmail.com}{mgmadsgram@gmail.com})  - 081293 - wtc324} \\
\Large{
  Hasanbasic, Mirza North (\href{mailto:pfl840@alumni.ku.dk}{pfl840@alumni.ku.dk}) - 040491 -pfl840}\\
\Large{
  Jørgensen, Mathias Bjørn (\href{mailto:jkf370@alumni.ku.dk}{jkf370@alumni.ku.dk}) - 230690- jkf370
  }
}
\title{
\vspace{3cm}
\Large{\nth{2} Group Assignment Report}
}

\begin{document}

\AddToShipoutPicture*{\put(0,0){\includegraphics*[viewport=0 0 700 600]{include/natbio-farve}}}
\AddToShipoutPicture*{\put(0,602){\includegraphics*[viewport=0 600 700 1600]{include/natbio-farve}}}

\AddToShipoutPicture*{\put(0,0){\includegraphics*{include/nat-en}}}

\clearpage\maketitle
\thispagestyle{empty}

\clearpage\newpage
\thispagestyle{plain}
\section*{Design}
Mads sidder og spiser chips hele dagen, dette giver ham energi til at være produktiv og lave kode.
Samtidig er Bjørn ``syg'' og Mirza er cheerleader!

Mads's designmønster i hvordan man skal opsætte det følger en slavisk tilgang.
Der er en header fil til definationer af makroer og en fil med koden.
\section*{What works}
Kravene for at Forwarding og Hazarding skal virke er opfyldt. Test metoderne er beskrevet nedenunder, men vi har også dobbelttjekket med hvad der står i beskrivelserne i .S filerne for at se om vi får det ønskede. Til Finalén, der virker: ORI, ANDI, ADDIU og LUI pga tidspres fik vi ikke indført de resterende.
Man kan bruge showstatus til at se om de rigtige værdier bliver printet korrekt.
\section*{Testing}
Igennem hele koden, tester vi retur koder for, at finde fejl der ellers ville ødelægge hele eksekveringen, og/eller sørge for, at man håndtere fejl passende.

Der er også en dump\_pregs() som printer værdierne ud, så man kan afluse sin kode. Samtidig med brug af getchar() kan man gå ignnem koden step-by-step og finde fejl.

Vi tester vores kode imod de udleverede test filer, ved brug af et shell script, [se test.sh].
Dette kan eksekveres med kommandoen ”sh test.sh” eller ./test.sh (dog skal man så huske at gøre den eksekverbar ”chmod +x test.sh”).


\nocite{Patterson:2013:COD:2568134}

\bibliography{mybib}
\bibliographystyle{ieeetr}

\end{document} 