% Document class: article with font size 11pt
\documentclass[11pt,a4paper,oneside]{report}
\usepackage[a4paper, hmargin={2.8cm, 2.8cm}, vmargin={2.5cm, 2.5cm}]{geometry}

\listfiles
\usepackage{mainPreamble}

\author{
  \Large{
    Hasanbasic, Mirza , (\href{mailto:pfl840@alumni.ku.dk}{pfl840@alumni.ku.dk}) - 081194 - pfl840
  }\\
  \Large{
    Jørgensen, Mathias Bjørn(\href{mailto:mathiashrytter@mail.com}{mathiashrytter@mail.com}) - 130793 - jkf370}\\
  \Large{
    Gram, Mads (\href{mailto:mgmadsgram@gmail.com}{mgmadsgram@gmail.com})  - 081293 - wtc324} \\
   \\
}
\title{
  \huge{Introduction to Compilers (OV)\\}
}


\begin{document}

\AddToShipoutPicture*{\put(0,0){\includegraphics*[viewport=0 0 700 600]{include/natbio-farve}}}
\AddToShipoutPicture*{\put(0,602){\includegraphics*[viewport=0 600 700 1600]{include/natbio-farve}}}

\AddToShipoutPicture*{\put(0,0){\includegraphics*{include/nat-en}}}

\clearpage\maketitle
\thispagestyle{empty}

\clearpage\newpage
\thispagestyle{plain}

%\overfullrule=2cm %Fjern til slut

%%%---
% Tekst her
%%%---

\section{Struktur i kode}
\subsection*{Implementation of boolean values}
To solve the exercise given in task one we have made changes to the following files \texttt{Lexer.lex}, \texttt{Parser.grm}, \texttt{Interpreter.sml}, \texttt{Typerchecker.sml}, and \texttt{CodeGen.sml}.


To start of we implemented boolean values \texttt{true} and \texttt{false}, which are defined as a keyword in the lexer.

\begin{code}[firstnumber=42]{Lexer.lex}
       | "true"         => Parser.TRUE pos
       | "false"        => Parser.FALSE pos
\end{code}

However inorder to interpret those keywords we must extend the parser to define those tokens.

\begin{code}[firstnumber=13]{Parser.grm}
%token <(int*int)> TRUE FALSE

\end{code}
\begin{code}[mathescape=false,firstnumber=68]{Parser.grm}
        | TRUE           { Constant (BoolVal true, $1) }
        | FALSE          { Constant (BoolVal false, $1) }
\end{code}
Where the second part assigns them to a boolean literal, to enable usage in other expressions.
Fasto can not use them yet due to MIPS not having a ``boolean type'', so we need to extend \texttt{CodeGen.sml} where we decided to assign \texttt{true} to the value 1 and \texttt{false} as 0. This will later be relevant when we create the \texttt{CodeGen.sml} for the $\&\&$ and $\|\|$ operators.
\begin{code}[firstnumber=171]{CodeGen.grm}
  | Constant (BoolVal b, pos) => if b = true then
        [Mips.LI(place, makeConst(1))]
          else if b = false then
        [Mips.LI(place, makeConst(0))]
          else raise Fail "Not a boolean value given"
\end{code}

\subsection*{Implementation of Times and Divide}
Due to the similarity in the implementation we will only look at divide.
The interesting difference in comparison with true and false in the \texttt{Lexer.lex} is that divide is not only represented as Divide but also represented a the symbol $\/$, this is handled with
\begin{code}[firstnumber=88]{Lexer.lex}
| `/`                 { Parser.DIVIDE (getPos lexbuf) }
\end{code}

A divide call takes three elements to call, the dividend, the divide symbol, and the divisor. The parser needs to understand these different elements, thise is done with
\begin{code}[firstnumber=77]{Parser.grm}
| Exp DIVIDE Exp { Divide($1, $3, $2)}
\end{code}
Where \$1 is the first expression, \$3 is the second expression, and \$2 is the position of divide.

In the parser we also need to handle the precedence of the call of divide. It needs to bind stronger than for example plus, so we add it to the precedence table in \texttt{Parser.grm}
\begin{code}[firstnumber=26]{Parser.grm}
%left PLUS MINUS
%left DIVIDE
\end{code}
This results in $x + y / z$ is interpreted as $x + (y / z)$

We check that the two given expressions are of the correct type, in regards to the operator, in this case it is a divide operation and therefore the expressions should be integers.

As seen we use the predefined function \texttt{CheckBinOp()} which evaluates the desired type against the evaluated types for each expressions.
\begin{code}[firstnumber=112]{TypeChecker.sml}
| In.Divide (e1, e2, pos)
  => let val (_, e1_dec, e2_dec) =
            checkBinOp ftab vtab (pos, Int, e1, e2)
     in (Int,
         Out.Divide (e1_dec, e2_dec, pos))
     end
\end{code}

We fist evaluate the two expressions, which then allows us to implement error handling when division by zero. Then the SML function \texttt{Int.quot} is used, the main difference between SML normal divide and \texttt{Int.quot} is that it rounds towards zero. This is done to align the interpreter with the generated MIPS code from \texttt{CodeGen.sml}.

\begin{code}[firstnumber=170]{Interpreter.sml}
| evalExp ( Divide(e1, e2, pos), vtab, ftab ) =
    let
      val res1   = evalExp(e1, vtab, ftab)
      val res2   = evalExp(e2, vtab, ftab)
    in
      if res2 = IntVal 0 then raise Fail "Division by zero"
        else
          case (res1, res2) of
            (IntVal n1, IntVal n2) => IntVal( Int.quot(n1, n2))
            | _ => invalidOperands
                "Divide on non-integral args: "
                    [(Int, Int)] res1 res2 pos
    end
\end{code}

The last part of implementing is divide, is the \texttt{CodeGen.sml}.
Here we create the registers and then evaluate the expressions and saves them in the before mentioned registers. With those to registers we can now call Mips.DIV and complete the calculation.
\begin{code}[firstnumber=251]{CodeGen.sml}
  | Divide (e1, e2, pos) =>
      let val t1 = newName "divide_L"
          val t2 = newName "divide_R"
          val code1 = compileExp e1 vtable t1
          val code2 = compileExp e2 vtable t2
      in  code1 @ code2 @ [Mips.DIV (place, t1 ,t2)]
      end
\end{code}
To show this is correct we look towards the tests for divide. Where the results for divide all evaluates to true
\begin{code}{divide.fo}
fun bool write_nl(bool b) =
  let res = write(b) in
  let tmp = write("\n") in
  res

fun bool main() =
    let x0 = write_nl(4 / 2 == 2) in
    let x1 = write_nl(2 / 4 == 0) in
    let x2 = write_nl(0 / 4 == 0) in
    write(x0 && x1 && x2)         //  printer w
\end{code}
And we get an error when we try to divide by 0, which is tested in \texttt{divideE.fo}

With this done we are now able to use divide.

\subsection*{And and Or}
As done in the previous section we will focus only on \texttt{AND}, as \texttt{OR} only differs slightly in how the short-circuiting is implemented.

In the \texttt{Lexer.lex}, \texttt{Parser.grm}, and \texttt{TypeChecker.sml} in the same fashion as \texttt{Divide}, so instead we look to the interpreter. Here the difference is that the function should be short-circuiting.
This results in us first evaluating the first expression, if it evaluates to \texttt{false} we are done and \texttt{AND} should return \texttt{false}.
\begin{code}[firstnumber=182]{Interpreter.sml}
| evalExp (And (e1, e2, pos), vtab, ftab) =
        let val r1 = evalExp(e1, vtab, ftab)
        in case r1 of
               BoolVal true => evalExp(e2, vtab, ftab)
             | BoolVal false  => BoolVal false
             | _ => raise Fail "First operand none boolean"
        end
\end{code}

The short-circuiting should also carry over into the \texttt{CodeGen.sml} where we need to recall that \texttt{true} and \texttt{false} are represented with the integers 1 and 0.
\begin{code}[firstnumber=413]{CodeGen.sml}
  | And (e1, e2, pos) =>
    let val t1 = newName "and_L"
        val t2 = newName "and_R"
        val falseLabel = newName "false"
        val code1 = compileExp e1 vtable t1
        val code2 = compileExp e2 vtable t2
    in
      code1 @
      [ Mips.LI (place,"0")
      , Mips.BEQ (t1, "0", falseLabel) ]
      @ code2 @
      [ Mips.BEQ (t2, "0", falseLabel)
      , Mips.LI (place, "1")
      , Mips.LABEL falseLabel ]
    end
\end{code}
As we can see in the code excerpt we solve this by creating the falseLabel, meaning if the left expression evaluates to \texttt{false} which is represented by 0, we can with Mips.BEQ jump to the falseLabel bypassing evaluating the second expression. Because we begin the code with loading 0 into our return register we then return \texttt{false} represented by 0. However if the first expression instead evaluates to \texttt{true} we then evaluate the second expression and if it is \texttt{true} we do not jump to the falseLabel and instead loads 1 into our return register.

First we run the test to check if \texttt{AND} is implemented correctly, we the test which is shown below. It tests out all possible combinations with two evaluations. Which correctly returns \texttt{true, false, false, false}.

\begin{code}{and.fo}
fun bool write_nl(bool b) =
  let res = write(b) in
  let tmp = write("\n") in
  res

fun bool main() =
    let x0 = write_nl(true and true) in
    let x1 = write_nl(true and false) in
    let x2 = write_nl(false and true) in
    write_nl(false and false)
\end{code}

Furthermore we use another test \texttt{short\_circuit.fo} which test if the short-circuiting is correctly implemented. This also test that \texttt{OR} is correctly short-circuited. Our implementation passes this test with grace.

\begin{code}{short\_circuit.fo}
fun bool no_way() = no_way()

fun bool main() =
  let a = write(false && no_way()) in
  let b = write(true || no_way()) in
  true
\end{code}

\subsection*{Not and Negate}

The \texttt{Lexer.lex} and \texttt{Interpreter.sml} is similar to the previous shown. The interesting comes when we are addressing the precedence level, since we want our program to interpret the different expression in the same order we general do. This means that when we write x * y + z == negate y or x == z we want it to be interpreted as (((x * y) + z) == (negate y)) or (x == z). In words we want it to first multiply x and y then add z, we then want it to negate y and see if it is equal to the previous calculation, afterwards see if x is equal to z, and finally if any of those two expressions are true.
\begin{code}[firstnumber=22]{Parser.grm}
%nonassoc ifprec letprec
%left OR
%left AND NOT
%left DEQ LTH NEGATE
%left PLUS MINUS
%left TIMES DIVIDE
\end{code}

For Not and Negate we have two expressions that only works on one type, boolean and integers respectively. This effects the \texttt{TypeChecker.sml} where we instead of using \texttt{checkBinOp} can merely evaluate the expression and check if is is the correct type.
\begin{code}[firstnumber=132]{code}
    | In.Not (e, pos)
      =>  let
            val (type_exp, e_dec) = checkExp ftab vtab e
          in
            if Bool = type_exp
            then (type_exp, Out.Not (e_dec, pos))
            else raise Error ("Not: Not bool", pos)
          end
\end{code}
With negate working in a similar fashion.

In the \texttt{CodeGen.sml} for Not we can use a smart Mips instruction to get the right result.
\begin{code}[firstnumber=258]{CodeGen.sml}
  | Not (e', pos) =>
      let
        val t1 = newName "Not"
        val code = compileExp e' vtable t1
      in
        code @ [Mips.XORI (place, t1, "1")]
      end
\end{code}
The reason this works is because we have two cases, one with \texttt{false} we compare with 1 which is equivalent to \texttt{true} resulting in \texttt{true}. The other scenario \texttt{true} which again is compared with 1, and Mips.XORI returns 0 equivalent to \texttt{false} when it evaluates two 1's.


Negate works by subtracting the number from zero, this makes the number negative, for a positive number and for a negative number it leads to a double negation which equals a positive number, for example to negate $42$ we say $0 - 42$ which equals $-42$.

\begin{code}[firstnumber=265]{CodeGen.sml}
  | Negate (e', pos) =>
    let
      val t1 = newName "Negate"
      val code = compileExp e' vtable t1
    in
      code @ [Mips.SUB (place, "0", t1)]
    end
\end{code}

\subsection*{Iota}
This was completely given by the \texttt{Task2.html} handout and will not be expanded on further.

\subsection*{Map}
In order to handle the anonymously functions call in map (and reduce) the \texttt{Lexer.lex} needed to be expanded with \texttt{fn} and $=>$ which we have denoted \texttt{FNEQ} in the \texttt{Parser.grm} they have been implemented in the same manner as earlier.
However the expression for map is a little more complicated. Map can be called in more than one way, both with a known function and an anonymous one (there is a third option that we have not implemented we will discuss later). To help with this we introduce FunArg to the parser.
\begin{code}[firstnumber=109]{code}
FunArg :  ID {FunName (#1 $1)}
        | FN Type LPAR RPAR FNEQ Exp
                        { Lambda ($2, [], $6, $1) }
        | FN Type LPAR Params RPAR FNEQ Exp
                        { Lambda ($2, $4, $7, $1) }
\end{code}
The first instance in FunArg is FunName which handles a known function, the other two instances are for the anonymous functions. This allow us to write a single version for map in the \texttt{Parser.grm}.
\begin{code}[firstnumber=86]{code}
        | MAP LPAR FunArg COMMA Exp RPAR
                         { Map ($3, $5, (), (), $1) }
\end{code}

The complexity with map also extends to the \texttt{TypeChecker.sml}. Where we need to check that our input is similar to the arguments that our functions uses.
\begin{code}[firstnumber=239]{TypeChecker.sml}
| In.Map (f, arr_exp, _, _, pos)
      =>
      let
          val (arr_type, arr_dec) = checkExp ftab vtab arr_exp
          val (fname, result_type, arg_types) = checkFunArg(f, vtab, ftab, pos)
          val element_type = case arr_type of
                                    Array (x) => x
                                  | _ => raise Error ("Map: input not an array", pos)
      in
          if element_type = hd arg_types
          then (Array result_type, Out.Map(fname, arr_dec, element_type, result_type, pos))
          else raise Error ("Map: wrong argument type " ^
                              ppType arr_type, pos)
      end
\end{code}

We are able to use \texttt{head (hd)} since \texttt{arg\_types} is a list of one element which is the type of the argument.

In the interpreter we first get the return type and the extract the array from the expression and use the SML List.map function in conjunction with the evaluated function, to get the result array.

\begin{code}[firstnumber=294]{Interpreter.sml}
  | evalExp ( Map (farg, arrexp, _, _, pos), vtab, ftab ) =
    let
      val return_type = rtpFunArg(farg, ftab, pos)
      val evaluated_arr = case evalExp(arrexp, vtab, ftab) of
        ArrayVal(arr, _) => arr
        | _ => raise Error("Map: Array expressions does not evaluate to ArrayVal", pos)
      val map_result = map(fn x => evalFunArg(farg, vtab, ftab, pos, [x])) evaluated_arr
    in
      ArrayVal(map_result, return_type)
    end
\end{code}

In the \texttt{CodeGen.sml} we decided to split the map function into the two cases, of the known function and the anonymous function. This is not a necessary step to take to solve the problem, as it can be seen in reduce where we only have one, however it was decided to take the approach to get a greater understanding of both cases. This does result in the two function being eerily similar, with the loop\_map call being the difference.
Firstly we focus on the map call with a known function. Map can deal with different kinds of input and output, in fact they are not necessarily the same, this is not problematic in a type checking sense, however when writing Mips code there is a difference in the amount of memory needed to store the different types, with integers taking 4 bytes and chars and booleans only using 1. When we load or store a result we then have to cases which we need to handle. To assist with this we have created a helper function called \texttt{mipsLoad()}, in addition to the existing ones.
\begin{code}[firstnumber=81]{CodeGen.sml}
fun mipsLoad elem_size = case elem_size of
                              One => Mips.LB
                            | Four => Mips.LW
\end{code}
The functions handles the issue of which mips instruction to use depending on type in the load scenario, the existing functions handles the rest of the problem.
In addition we also have use of several registers, with the most significant ones being addr\_reg, where we load our arguments, and result\_reg where we store our result, in addition to those we have several registers used temporarily to store our values. In mips we can not change the size of our arrays dynamically so we need to know the size for our result\_ reg, which should be identically to our argument array. Here we exploit the fact that when we create our argument array the first argument is the size of the array, which we then store in size\_reg, and with the use of the helper function dynalloc create the result\_reg.

\begin{code}[firstnumber=538]{CodeGen.sml}
        val addr_reg  = newName "addr_reg"
        val addr_code = compileExp arr_exp vtable addr_reg

        val size_reg  = newName "size_reg"
        val size_code = [Mips.LW(size_reg, addr_reg, " 0")]

        val result_reg = newName "result_reg"
\end{code}

In this instance both our registers addr\_reg and result\_reg both contain the pointers to the size of their arrays and not the fist element of their arrays so we need to initialise our registers to begin our actual calculations.
\begin{code}[firstnumber=546]{CodeGen.sml}
        val i_reg = newName "i_reg"
          val init_regs = [ Mips.ADDI (result_reg, place, "4")
                          , Mips.ADDI (addr_reg, addr_reg, "4")
                          , Mips.MOVE (i_reg, "0") ]
\end{code}

Since the size of arrays are both integers we do not need to concern ourselves with type size here.

Since we can not calculate with pointers we have to load the actual value, which we then use the helper function \texttt{applyRegs()} to call the actual function. The value is then stored in our result\_reg.
\begin{code}[firstnumber=561]{CodeGen.sml}
        val load_arg = [mipsLoad elem_size (tmp2_reg, addr_reg, "0")]

        val loop_map = applyRegs(farg, [tmp2_reg], tmp3_reg, pos)

        val save_res = [mipsStore ret_size (tmp3_reg, result_reg, "0") ]
\end{code}
Implied by the name loop\_map this only solves one instance of the problem, so we need a loop to run over the entire argument array. This is done in the same manner as \texttt{iota} with the one significant difference. We have to increment our addr\_reg and result\_reg with correct amount, as dictated by the type of variable.

\begin{code}[firstnumber=567]{CodeGen.sml}
  Mips.ADDI (addr_reg, addr_reg, makeConst (elemSizeToInt elem_size))
, Mips.ADDI (result_reg, result_reg, makeConst (elemSizeToInt ret_size))
\end{code}
Where makeConst (elemSizeToInt ...) refers to
\begin{code}[firstnumber=535]{CodeGen.sml}
val elem_size = getElemSize elem_type
val ret_size  = getElemSize ret_type
\end{code}
Which all adds up to
\begin{code}[firstnumber=573]{CodeGen.sml}
in addr_code
          @ size_code
          @ dynalloc (size_reg, place, ret_type)
          @ init_regs
          @ loop_header
          @ load_arg
          @ loop_map
          @ save_res
          @ loop_footer
        end
\end{code}

Which allows us to call map on known functions, however for anonymous functions we can not call the \texttt{applyRegs()} function. Instead we have to extend our vtable with the parameters given in the anonymous function to be able to evaluate the body of the function.
\begin{code}[firstnumber=612]{CodeGen.sml}
  val vtable' = case params of
                  Param(Pname, Ptype)::[]   => SymTab.bind Pname tmp2_reg vtable
                | _ => raise Error ("Error no paramater name",pos)

  val loop_map = compileExp body vtable' tmp3_reg
\end{code}

This then allows us to call map with anonymous functions as well as known functions. However we have not handled the \texttt{op+(-,*,/)} version. We have not implemented this due to poor time management.
In order to implement it would be necessary to add the four cases to the \texttt{Lexer.lex}, or just add op as a keyword and use it in combination with the existing tokens.
\begin{code}{Code Example}
       | "op"         => Parser.OP pos
\end{code}
In the \texttt{Parser.grm} it would be needed to expand the FunArg to handle this
\begin{code}{Code Example}
FunArg :  ID {FunName (#1 $1)}
        | OP Exp
             { OP( \$2 ) }
\end{code}
For the type checker and interpreter we would need to determine if the case can be handled by the helper functions or additional has to be written. Whereas for the \texttt{CodeGen.sml} we would create an additional map to handle the case of \texttt{op+(-,*,/)}.


To show that our solution works for known and anonymous function we can run the \texttt{anonymous map.fo} and \texttt{inline\_map.fo} tests, which returns the desired results.


\subsection*{Reduce}
While reduce is slighter different than map, it is insignificantly for both the \texttt{Lexer.lex} and \texttt{Parser.grm} however in the \texttt{TypeChecker.sml} we have a major chance in comparison to \texttt{map}. Where map can chance type during the function call \texttt{reduce} need to return the same type as the argument it takes. It also takes an additional argument than the list its evaluating, which should be the same as well.
\begin{code}[firstnumber=253]{TypeChecker.sml}
| In.Reduce (f, n_exp, arr_exp, _, pos)
      =>
      let val (n_type, n_dec) = checkExp ftab vtab n_exp
          val (arr_type, arr_dec) = checkExp ftab vtab arr_exp
          val (fname, result_type, arg_types) = checkFunArg(f, vtab, ftab, pos)
          val arr_type = case arr_type of
                      Array(x) => x
                    | _ => raise Error ("Reduce: wrong argument type",pos)
      in
          if (n_type = arr_type andalso n_type = hd arg_types)
          then (result_type, Out.Reduce(fname, n_dec, arr_dec, n_type, pos))
          else raise Error ("Reduce: wrong argument type", pos)
      end
\end{code}

In the \texttt{Interpreter.sml} is very similar to \texttt{map}, however to evaluate the function we instead use the SML function foldl, and it evaluates to a single value.
\begin{code}[firstnumber=305]{Interpreter.sml}
  | evalExp ( Reduce (farg, ne, arrexp, tp, pos), vtab, ftab ) =
    let
      val return_type = rtpFunArg(farg, ftab, pos)
      val eval_n = evalExp (ne, vtab, ftab)
      val eval_arr = case evalExp(arrexp, vtab, ftab) of
          ArrayVal (arr, _) => arr
        | _ => raise Error("Reduce: array exp doesnot eval to ArrayVal", pos)
      val reduce_result = foldl(fn (x,y) => evalFunArg(farg, vtab, ftab, pos, [x,y])) eval_n eval_arr
    in
      reduce_result
    end
\end{code}

The fact that it evaluates to a single value, can also be exploited in the \texttt{CodeGen.sml}. Which means we do not need to store after each iteration of our loop, we can instead carry the value over in a temp reg and only load the new value from the argument array. We just have to remember to move our result from the temporary register after the loop has completed.
\begin{code}[firstnumber=662]{CodeGen.sml}
val funArgCodeGen = case binop of
              FunName f => applyRegs(f, [acc_reg, tmp2_reg], acc_reg, pos)
            | Lambda (rettype, params, body, _) => (case params of
                   (Param(xName, xType)::Param(yName, yType)::[]) =>
                          compileExp body (SymTab.bind yName tmp2_reg (SymTab.bind xName acc_reg vtable)) acc_reg
                 | _ => raise Error ("Error no paramater name",pos))

val loop_reduce = [ mipsLoad elem_size (tmp2_reg, addr_reg, "0") ]
                  @ funArgCodeGen
\end{code}

This does leave one problem compared to map, when we have completed the iteration our value is saved in the wrong place, so we need to alter our loop\_footer.
\begin{code}[firstnumber=672]{CodeGen.sml}
        val loop_footer = [ Mips.ADDI (addr_reg, addr_reg, makeConst (elemSizeToInt elem_size))
                            , Mips.ADDI (i_reg, i_reg, "1")
                            , Mips.J loop_beg
                            , Mips.LABEL loop_end
                            , Mips.MOVE(place, acc_reg)
                            ]
\end{code}
Also worth noting is there is no result\_reg to increment since we keep updating acc\_reg. When the loop has terminated we simply move our value and the code terminates.
\begin{code}[firstnumber=678]{CodeGen.sml}
        in addr_code
          @ acc_code
          @ size_code
          @ init_regs
          @ loop_header
          @ loop_reduce
          @ loop_footer
        end
\end{code}

Unlike \texttt{map} we only have a single \texttt{reduce} function, which means we handle the difference known and anonymous functions in a case in the loop. Where we for known functions simply use \texttt{applyRegs()} as in maps. For the anonymous functions we need instead need to extend the vtable with two variables, since \texttt{reduce} takes two arguments. After which we proceed as in maps and evaluate the body of the function over the new vtable.

Running the test \texttt{reduce.fo} we confirm it works for a known function, and for the anonymous functions we use anon\_reduce.fo to show the correctness.

\subsection*{Constant Folding}
In \texttt{Times} we made the following rules
\begin{code}[firstnumber=43]{CopyConstPropFold.sml}
      | Times (e1, e2, pos) =>
        let val e1' = copyConstPropFoldExp vtable e1
            val e2' = copyConstPropFoldExp vtable e2
        in case (e1', e2') of
          (Constant (IntVal x, _), Constant (IntVal y, _)) => Constant (IntVal (x*y), pos)
          | (Constant                 (IntVal 1, _), _)   =>
            e2' (* Multiplication by one x * 1 = x *)
          | (Constant (IntVal ~1, _), _)   =>
            Negate(e2', pos)          (* Multiplication by minus one x * -1 = -x *)
          | (_, Constant (IntVal 1, _))   =>
            e1'                       (* Multiplication by one x * 1 = x *)
          | (_, Constant (IntVal ~1, _))   =>
            Negate(e1', pos)          (* Multiplication by minus one x * -1 = -x *)
          | _                             =>
            Times(e1', e2', pos)
        end
\end{code}
Which equates to the two following multiplications rules
\begin{center}
  \begin{verbatim}
    x * 1 = x
    x * (-1) = -x
    \end{verbatim}
\end{center}

In \texttt{AND} we wrote
\begin{code}[firstnumber=59]{CopyConstPropFold.sml}
      | And (e1, e2, pos) =>
        let val e1' = copyConstPropFoldExp vtable e1
            val e2' = copyConstPropFoldExp vtable e2
        in case (e1', e2') of
               (Constant (BoolVal a, _), Constant (BoolVal b, _)) =>
               Constant (BoolVal (a andalso b), pos)
             | _ => And (e1', e2', pos)
        end
\end{code}
Which in case of both expressions being BoolVals it reduces it to a single expression.

In \texttt{Divide} we added
\begin{code}[firstnumber=139]{CopyConstPropFold.sml}
      | Divide (e1, e2, pos) =>
        let val e1' = copyConstPropFoldExp vtable e1
            val e2' = copyConstPropFoldExp vtable e2
        in case (e1', e2') of
               (Constant (IntVal x, _), Constant (IntVal y, _)) =>
                 (Constant (IntVal (Int.quot (x,y)), pos)
                   handle Div => Divide (e1', e2', pos))
              | (_, Constant (IntVal 1, _)) =>
                e1' (* Division by minus one x / 1 = x *)
              | (_, Constant (IntVal ~1, _)) =>
                Negate(e1', pos) (* Division by minus one x / -1 = -x *)
              | _ => Divide (e1', e2', pos)
        end
\end{code}
We added division by 1 being irrelevant, and division by negative on simply negates the expression.


\subsection*{Copy/Constant Propagation}

We were not able to complete this part of the exercise, since we were uncertain of what precisely was meant to be done in this part.

\end{document}