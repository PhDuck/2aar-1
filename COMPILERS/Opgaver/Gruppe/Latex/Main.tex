% Document class: article with font size 11pt
\documentclass[11pt,a4paper,oneside]{report}

\listfiles
\usepackage{mainPreamble}


\begin{document}

%\overfullrule=2cm %Fjern til slut

\setcounter{page}{0}
\renewcommand\thepage{\roman{page}}
\begin{titlepage}
    \begin{center}
        \vspace*{1cm}

        \Huge
        \textbf{101: Byg egen compiler}

        \vspace{0.5cm}
        \LARGE
        undertitel

        \vspace{1.5cm}

        \textbf{Mathias}\\
        \textbf{Mirza}\\
        \textbf{Mads }\\

        \vspace{1.5cm}

        Gruppemedlemmere:\\

        \vspace{0.5cm}



        \vfill

        Gruppe Opgave

        \vspace{0.8cm}

        \includegraphics[width=0.4\textwidth]{sam1}

        \Large
        Datalogisk Institut\\
        Compilers 2015

    \end{center}
\end{titlepage} 


\tableofcontents

\newpage

\setcounter{page}{1}
\renewcommand\thepage{\arabic{page}}

%%%---
% Tekst her
%%%---

For at løse opgaver i task 1, kigger vi kun på \texttt{Lexer.lex}, \texttt{Parser.grm}, \texttt{Interpreter.sml}, \texttt{Typerchecker.sml} og \texttt{codegen.sml}.

Hvis vi, tager det slavisk, som i den rækkefølge de skal løses i. Det vil sige, for at implentere \texttt{TRUE} og \texttt{FALSE}, så skal de først defineres i \texttt{Lexer.lex} under keywords, der indsættes
\begin{lstlisting}[firstnumber=42]
       | "true"         => Parser.TRUE pos
       | "false"        => Parser.FALSE pos
\end{lstlisting}
i \texttt{Parser.grm} er begge blevet implementeret som tokens og expresseion
\begin{lstlisting}[firstnumber=13]
%token <(int*int)> TRUE FALSE
\end{lstlisting}
\begin{lstlisting}[mathescape=false,firstnumber=68]
        | TRUE           { Constant (BoolVal true, $1) }
        | FALSE          { Constant (BoolVal false, $1) }
\end{lstlisting}
Som vi gjorde med \texttt{TRUE} og \texttt{FALSE}, så skal der også defineres en rule token for \texttt{TIMES} og \texttt{DIVIDE} inde i \texttt{Lexer.lex}, og der laves det samme som med \texttt{TRUE}. Det samme gælder for selve token, hvor det er en \texttt{(int*int)}. Selve precedence level af \texttt{TIMES} og \texttt{DIVIDE} er den højste, dvs. at $4 + 2 \cdot 3$ betyder $4 + (2 \cdot 3)$


%HUSK: Laver hjælpe funktion til task 2: Mips load i Codegen linie 80ish

%%%--- tekst    
% Tekst her
%%%---


\clearpage
\addcontentsline{toc}{chapter}{Litteratur}
\bibliographystyle{plain}
\bibliography{mainBib}


\end{document} 