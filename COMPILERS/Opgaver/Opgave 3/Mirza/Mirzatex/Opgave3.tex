
% Document class: article with font size 11pt
% ---------------
\documentclass[11pt,a4paper]{report}

% Call packages
% ---------------
\usepackage{comment} %Possible to comment larger sections
%http://get-software.net/macros/latex/contrib/comment/comment.pdf
\usepackage[T1]{fontenc} %oriented to output, that is, what fonts to use for printing characters.
\usepackage[utf8]{inputenc} %allows the user to input accented characters directly from the keyboard
%Put fontenc before inputenc
\usepackage{fourier} %Better typeface
%http://mirrors.dotsrc.org/ctan/fonts/fourier-GUT/doc/latex/fourier/fourier-doc-en.pdf
\usepackage[danish]{babel}														     % Danish
\usepackage[protrusion=true,expansion=true]{microtype}				                 % Better typography
%http://www.khirevich.com/latex/microtype/
\usepackage{amsmath,amsfonts,amsthm, amssymb}							 % Math packages
\usepackage[pdftex]{graphicx} %puts to pdf and graphic
%http://www.kwasan.kyoto-u.ac.jp/solarb6/usinggraphicx.pdf
\usepackage{xcolor,colortbl}
%http://mirrors.dotsrc.org/ctan/macros/latex/contrib/xcolor/xcolor.pdf
%http://texdoc.net/texmf-dist/doc/latex/colortbl/colortbl.pdf
\usepackage{tikz} %documentation http://www.ctan.org/pkg/pgf
\usetikzlibrary{matrix}
\usepackage{parskip} %http://www.ctan.org/pkg/parskip
%http://tex.stackexchange.com/questions/51722/how-to-properly-code-a-tex-file-or-at-least-avoid-badness-10000
%Never use \\ but instead press "enter" twice. See second website for more info
\usepackage{natbib}
%\usepackage{tablefootnote}
%Brug H for figur lige HER!!
\usepackage{float}
%\caption{} til tabeller
\usepackage{caption}

\usepackage{pbox}

%Skal ikke bruges i slut
\usepackage[draft,danish]{fixme} %Danish latex book
%\usepackage{showkeys} %Brug kun til at holde styr på labels

%%--------------------------------
%Til at vise links i pdf filen så man kan hoppe frem og tilbage
\usepackage{hyperref}
\hypersetup{
    colorlinks,
    linkcolor={blue!35!black},
    citecolor={blue!50!black},
    urlcolor={blue!80!black}
}

\usepackage{fancyhdr} % Required for custom headers
\usepackage{lastpage} % Required to determine the last page for the footer
\usepackage{extramarks} % Required for headers and footers
\usepackage{listings} % Required for insertion of code
\usepackage{courier} % Required for the courier font
\usepackage{lipsum} % Used for inserting dummy 'Lorem ipsum' text into the template


\usepackage[top=1in, bottom=1.25in, left=1.25in, right=1.25in, footskip=0.25in]{geometry}

\linespread{1.15} % Line spacing

% Set up the header and footer
\pagestyle{fancy}
\lhead{\hmwkAuthorName} % Top left header
\chead{\hmwkClass\: \hmwkTitle} % Top center head
\rhead{\firstxmark} % Top right header
\lfoot{\lastxmark} % Bottom left footer
\cfoot{} % Bottom center footer
\rfoot{Page\ \thepage\ of\ \protect\pageref{LastPage}} % Bottom right footer
\renewcommand\headrulewidth{0.4pt} % Size of the header rule
\renewcommand\footrulewidth{0.4pt} % Size of the footer rule

\setlength\parindent{0pt} % Removes all indentation from paragraphs

%----------------------------------------------------------------------------------------
%	DOCUMENT STRUCTURE COMMANDS
%	Skip this unless you know what you're doing
%----------------------------------------------------------------------------------------

% Header and footer for when a page split occurs within a problem environment
\newcommand{\enterProblemHeader}[1]{
\nobreak\extramarks{#1}{#1 continued on next page\ldots}\nobreak
\nobreak\extramarks{#1 (continued)}{#1 continued on next page\ldots}\nobreak
}

% Header and footer for when a page split occurs between problem environments
\newcommand{\exitProblemHeader}[1]{
\nobreak\extramarks{#1 (continued)}{#1 continued on next page\ldots}\nobreak
\nobreak\extramarks{#1}{}\nobreak
}

\setcounter{secnumdepth}{0} % Removes default section numbers
\newcounter{homeworkProblemCounter} % Creates a counter to keep track of the number of problems

\newcommand{\homeworkProblemName}{}
\newenvironment{homeworkProblem}[1][Task \arabic{homeworkProblemCounter}]{ % Makes a new environment called homeworkProblem which takes 1 argument (custom name) but the default is "Problem #"
\stepcounter{homeworkProblemCounter} % Increase counter for number of problems
\renewcommand{\homeworkProblemName}{#1} % Assign \homeworkProblemName the name of the problem
\section{\homeworkProblemName} % Make a section in the document with the custom problem count
\enterProblemHeader{\homeworkProblemName} % Header and footer within the environment
}{
\exitProblemHeader{\homeworkProblemName} % Header and footer after the environment
}

\newcommand{\problemAnswer}[1]{ % Defines the problem answer command with the content as the only argument
\noindent\framebox[\columnwidth][c]{\begin{minipage}{0.98\columnwidth}#1\end{minipage}} % Makes the box around the problem answer and puts the content inside
}

\newcommand{\homeworkSectionName}{}
\newenvironment{homeworkSection}[1]{ % New environment for sections within homework problems, takes 1 argument - the name of the section
\renewcommand{\homeworkSectionName}{#1} % Assign \homeworkSectionName to the name of the section from the environment argument
\subsection{\homeworkSectionName} % Make a subsection with the custom name of the subsection
\enterProblemHeader{\homeworkProblemName\ [\homeworkSectionName]} % Header and footer within the environment
}{
\enterProblemHeader{\homeworkProblemName} % Header and footer after the environment
}

%----------------------------------------------------------------------------------------
%	NAME AND CLASS SECTION
%----------------------------------------------------------------------------------------

\newcommand{\hmwkTitle}{Assignment\ \#3} % Assignment title
\newcommand{\hmwkDueDate}{Sunday,\ December\ 3,\ 2016} % Due date
\newcommand{\hmwkClass}{Compilers} % Course/class
\newcommand{\hmwkClassTime}{Task 2} % Class/lecture time
\newcommand{\hmwkClassInstructor}{Genafl:} % Teacher/lecturer
\newcommand{\hmwkAuthorName}{Mirza Hasanbasic} % Your name

%----------------------------------------------------------------------------------------
%	TITLE PAGE
%----------------------------------------------------------------------------------------

\title{
\vspace{2in}
\textmd{\textbf{\hmwkClass:\ \hmwkTitle}}\\
\normalsize\vspace{0.1in}\small{Due\ on\ \hmwkDueDate}\\
\vspace{0.1in}\large{\textit{\hmwkClassInstructor\ \hmwkClassTime}}
\vspace{3in}
}

\author{\textbf{\hmwkAuthorName}}
\date{} % Insert date here if you want it to appear below your name


%----------------------------------------------------------------------------------------
%	MATH
%----------------------------------------------------------------------------------------
%\newcommand{\Real}{\mathbb R}
%\newcommand{\Complex}{\mathbb C}
%\newcommand{\Field}{\mathbb F}
%\newcommand{\RPlus}{[0,\infty)}
%%
%\newcommand{\norm}[1]{\left\Vert#1\right\Vert}
%\newcommand{\essnorm}[1]{\norm{#1}_{\text{\rm\normalshape ess}}}
%\newcommand{\abs}[1]{\left\vert#1\right\vert}
%\newcommand{\set}[1]{\left\{#1\right\}}
%\newcommand{\seq}[1]{\left<#1\right>}
%\newcommand{\eps}{\varepsilon}
%\newcommand{\To}{\longrightarrow}
%\newcommand{\RE}{\operatorname{Re}}
%\newcommand{\IM}{\operatorname{Im}}
%\newcommand{\Poly}{{\cal{P}}(E)}
%\newcommand{\EssD}{{\cal{D}}}
%% THEOREMS ----------------------------------------------------------------
%\theoremstyle{plain}
%\newtheorem{thm}{Theorem}[section]
%\newtheorem{cor}[thm]{Corollary}
%\newtheorem{lem}[thm]{Lemma}
%\newtheorem{prop}[thm]{Proposition}
%%
%\theoremstyle{definition}
%\newtheorem{defn}{Definition}[section]
%%
%\theoremstyle{remark}
%\newtheorem{rem}{Remark}[section]
%%
%\numberwithin{equation}{section}
%\renewcommand{\theequation}{\thesection.\arabic{equation}}


\lstset{
  frame=single,
  numbers=left,
  mathescape,
  literate={->}{$\rightarrow$}{2}
           {ε}{$\varepsilon$}{1}
           {=>}{$\Rightarrow$}{2}
}

\begin{document}

\maketitle

%----------------------------------------------------------------------------------------
%	TABLE OF CONTENTS
%----------------------------------------------------------------------------------------

%\setcounter{tocdepth}{1} % Uncomment this line if you don't want subsections listed in the ToC

\newpage
\tableofcontents
\newpage

%----------------------------------------------------------------------------------------
%	PROBLEM 1
%----------------------------------------------------------------------------------------

% To have just one problem per page, simply put a \clearpage after each problem


\begin{homeworkProblem}
Det skal siges, at jeg har vedhæftet en .txt fil, så du lettere kan afprøve koden som er skrevet. Det der står i dokumentet er til det visuelle.

\section*{\textbf{a})}

Vi har

\begin{lstlisting}
vtable = [a -> v, b -> w];

while (b != 0) && (a/b != 0)
	if b < a then {a := a - b}
			 else {b := b - a}
\end{lstlisting}

Hvor intermediate koden er

\begin{lstlisting}
t_0 = v
t_1 = w
LABEL LoopStart
IF t_1 != 0 then NEXT0 else END (Brug rigtig syntax i tex filen !=)
LABEL NEXT0
t_2 = t_0 mod t_1
IF t_2 != 0 then NEXT1 else END
LABEL NEXT1
t_3 = t_1 - t_0
IF t_3 < 0 then NEXT2 else NEXT3
LABEL NEXT2
t_0 = t_0 - t_1
GOTO LoopStart
LABE NEXT3
t_1 = t_1 - t_0
GOTO LoopStart
LABEL END
\end{lstlisting}

og MIPS koden vil være

\newpage
\begin{lstlisting}[mathescape=false]
.data
	a: .word 8
	b: .word 33
.text
main:
lw $t0, a			\# load 8
lw $t1, b			\# load 33
LoopStart:			\# LABEL
beq $t1, $0, END	        \# Checking if t1 == 0
div $t0, $t1			\# dividing to get modulus
mfhi $t2			\# Getting the remainder, moving to $t2
beq $t2, $0, END		\# checking if t2 == 0
sub $t3, $t1, $t0		\# t3 = t1 - t0
bgez $t3, ELSE			\# t3 >= 0
sub $t0, $t0, $t1		\# first then statement a = a - b
j LoopStart			\# jumping to loopstart
ELSE:				   \# Now else statement
sub $t1, $t1, $t0		\# b = b - a
j LoopStart
END:
				\# tinyurl.com/neve79o
li $v0, 1		     \# printer udregnet variable ud.
add $a0, $t0, $zero		
syscall					

li $v0, 11				
li $a0, 10				
syscall					

li $v0, 1				
add $a0, $t1, $zero		
syscall					  			
\end{lstlisting}

\section*{\textbf{b})}

\begin{lstlisting}
Li t0, x
Li t1, y
Li t2, 1
Slt t3, t1, t0
Slt t4, t3, t2
\end{lstlisting}

\end{homeworkProblem}

\begin{homeworkProblem}
\section*{\textbf{a})}

\begin{table}[H]
\centering
\begin{tabular}{|l|l|l|l|}
\hline
i  & succ{[}i{]} & gen{[}i{]} & kill{[}i{]} \\ \hline
1  & 2           &            &             \\ \hline
2  & 7,3         & a,b        &             \\ \hline
3  & 4           &            &             \\ \hline
4  & 5           & a          & t           \\ \hline
5  & 6           & b          & a           \\ \hline
6  & 7           & t          & b           \\ \hline
7  & 8           &            &             \\ \hline
8  & 9           &            & z           \\ \hline
9  & 10          & b,a        & b           \\ \hline
10 & 1,11        & b,z        &             \\ \hline
11 &             &            &             \\ \hline
12 &             &    a        &             \\ \hline
\end{tabular}
\end{table}

\section*{\textbf{b})}

\textcolor[rgb]{0.76,0.00,0.02}{\textbf{FIX:}}

\begin{table}[H]
\centering
\begin{tabular}{|l||l|l||l|l||l|l|l|l|}
\hline
\multicolumn{3}{|c||}{Initial} & \multicolumn{2}{c||}{Iteration 1} & \multicolumn{2}{c|}{Iteration 2} & \multicolumn{2}{c|}{Iteration 3} \\ \hline
i   & out{[}i{]}  & in{[}i{]} & out{[}i{]}      & in{[}in{]}     & out{[}i{]}      & in{[}in{]} & out{[}i{]}      & in{[}in{]}     \\ \hline
1   &             &           &     a,b         &    a,b         &    a,b          &   a,b      &    a,b          &   a,b        \\ \hline
2   &             &           &     a,b         &    a,b         &    a,b          &   a,b      &    a,b          &   a,b       \\ \hline
3   &             &           &     a,b         &    a,b         &    a,b          &   a,b      &    a,b          &   a,b    \\ \hline
4   &             &           &     b,t         &    a,b         &    b,t          &   a,b      &    b,t          &   a,b           \\ \hline
5   &             &           &     a,t         &    b,t         &    a,t          &   b,t      &    a,t          &   b,t    \\ \hline
6   &             &           &     a,b         &    a,t         &    a,b          &   a,t      &    a,b          &   a,t    \\ \hline
7   &             &           &     a,b         &    a,b         &    a,b          &   a,b      &    a,b          &   a,b    \\ \hline
8   &             &           &     a,b,z       &    a,b         &    a,b,z        &   a,b      &    a,b,z        &   a,b     \\ \hline
9   &             &           &     b,z,a       &    a,b,z       &    a,b,z        &   a,b,z    &    a,b,z        &   a,b,z   \\ \hline
10  &             &           &        a      &    a, b,z      &    a,b          &   a,b,z    &    a,b          &   a,b,z     \\ \hline
11  &             &           &        a        &      a         &       a         &      a     &       a         &      a    \\ \hline
12  &             &           &                 &    a           &                 &    a       &                 &    a     \\ \hline
\end{tabular}
\end{table}

\section*{\textbf{c})}

\begin{table}[H]
\centering
\begin{tabular}{|c|c|c|}
  \hline
  % after \\: \hline or \cline{col1-col2} \cline{col3-col4} ...
  i & left & interferes with \\ \hline
  4 & t & a,b \\ \hline
  5 & a & b,t \\ \hline
  6 & b & a,t \\ \hline
  8 & z & a,b \\ \hline
  9 & b & a,z \\ \hline
\end{tabular}
\end{table}

\[
\begin{tikzpicture}[-, node distance=40mm]
    \node (start)                           {A};
    \node (below) [below of = start]        {Z};
    \node (right) [right of = start]        {B};
    \node (belowRight) [below of = right]   {T};

    \path (start) edge node {} (below)
          (start) edge node {} (belowRight)
          (start) edge node {} (right)
          (right) edge node {} (belowRight)
          (right) edge node {} (below);
\end{tikzpicture}
\]

\section*{\textbf{d})}

\textcolor[rgb]{0.76,0.00,0.02}{\textbf{FIX:}}

\begin{table}[H]
\centering
\begin{tabular}{|c|c|c|}
  \hline
  % after \\: \hline or \cline{col1-col2} \cline{col3-col4} ...
  node & Neighbours & color \\ \hline
  a &  & 1 \\ \hline
  b & a & 2 \\ \hline
  t & a, b & 3 \\ \hline
  z & a,b & 3 \\ \hline
\end{tabular}
\end{table}


\[
\begin{tikzpicture}[-, node distance=40mm]
    \node (start)                           {\color{red}A};
    \node (below) [below of = start]        {\color{blue}Z};
    \node (right) [right of = start]        {\color{green}B};
    \node (belowRight) [below of = right]   {\color{blue}T};

    \path (start) edge node {} (below)
          (start) edge node {} (belowRight)
          (start) edge node {} (right)
          (right) edge node {} (belowRight)
          (right) edge node {} (below);
\end{tikzpicture}
\]

\section*{\textbf{e})}


\[
\begin{tikzpicture}[-, node distance=40mm]
    \node (start)                           {A};
    \node (below) [below of = start]        {\color{blue}Z};
    \node (right) [right of = start]        {\color{green}B};
    \node (belowRight) [below of = right]   {\color{blue}T};

    \path (start) edge node {} (below)
          (start) edge node {} (belowRight)
          (start) edge node {} (right)
          (right) edge node {} (belowRight)
          (right) edge node {} (below);
\end{tikzpicture}
\]

\begin{lstlisting}
gcd(a,b) {
    M[address$_a$] := a
    LABEL start
    a$_i$ := M[address$_a$]
    IF a$_i$ < b THEN next ELSE swap
    LABEL swap
    a$_i$ := M[address$_a$]
    t := a$_i$
    a$_i$ := b
    M[address$_a$] = a$_i$
    b := t
    LABEL next
    z := 0
    a$_i$ = M[address$_a$]
    b := b mod a$_i$
    IF b = z THEN end ELSE start
    LABEL end
    a := M[address$_a$]
    RETURN a
}
\end{lstlisting}

og dette vil være med 2 registre

\textcolor[rgb]{0.76,0.00,0.02}{\textbf{FIX:}}

\[
\begin{tikzpicture}[-, node distance=40mm]
    \node (start)                           {\color{red}A};
    \node (below) [below of = start]        {Z};
    \node (right) [right of = start]        {\color{green}B};
    \node (belowRight) [below of = right]   {T};

    \path (start) edge node {} (below)
          (start) edge node {} (belowRight)
          (start) edge node {} (right)
          (right) edge node {} (belowRight)
          (right) edge node {} (below);
\end{tikzpicture}
\]

\end{homeworkProblem}


\end{document} 